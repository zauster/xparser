\section{Memory}

Allocation of memory for agents and messages should be in one continuous area of
memory. This is so that the command \textit{sizeof} can return the byte size of
agent and message memory in one execution. This is important when sending agents
and messages in parallel using MPI. The main problem is the use of dynamic arrays
because the associated memory allocated is not going to be apart of the
continuous memory allocated for agents and messages. It is possible to handle this but it is
expensive. Because messages are continuously being sent and received they should
not contain any dynamic arrays. It is only when agents are being load balanced
that they need to be sent in parallel.

Currently user-defined data types are allocated as pointers in agent memory but
this has been modified in a new version to be released. This means that instead
of user using an arrow '-\textgreater' to dereference variables, a dot '.' is
used to access the data structure.

Dynamic array data structures are also not allocated as a pointer (but the actual
dynamic array is) which means functions to interact with a dynamic array data
structure need to pass a pointer. This means the use of the ampersand '\&' to
reference the data structure.

\subsection{Agent Memory Management}

Each agent has an associated memory data structure. Since early versions of the
framework all agents have been managed in one list. This was so that the list
could be randomised and therefore remove any artifacts of agents having priority
over other agents by always being executed first. In essence the only effect this
has is to randomise the message output and therefore the message input to agents.
The current framework has a generic agent memory stucture that can point to any
specific agent type.

With the introduction of the new message board library the action of randomising
(or now also sorting and filtering) messages the need to randomise the agent list
is redundant. Also redundant is the need to have a single list of all the agents.
The generic agent memory structure is therefore not needed and each agent type
can have it's only seperate list.
