\section{Model Implementation}\label{model_implementation}

If a separate function file is written for every agent, then the
parser would be generating a header file for every agent. This has
to be included for processing. Thus, at the top of each file two
headers are required:

\begin{mylisting}
\begin{verbatim}
#include "header.h"
#include "<agentname>_agent_header.h"
\end{verbatim}
\end{mylisting}

Agent functions can be defined as a list of functions in the
function file such as:

\begin{mylisting}
\begin{verbatim}
int function_name()
{
   /* Function code here */

   return 0; /* Returning zero means the agent is not removed */
}
\end{verbatim}
\end{mylisting}

\subsection{Accessing Agent Memory Variables}

After including the specific agent header, the variables in the
agent memory can be accessed by capitalising the variable name.

\begin{mylisting}
\begin{verbatim}
MY_SINGLE_VARIABLE
\end{verbatim}
\end{mylisting}

To access elements of a static array just use square brackets and index number as normal.

\begin{mylisting}
\begin{verbatim}
MY_STATIC_ARRAY[index]
\end{verbatim}
\end{mylisting}

To access the elements and the size of dynamic array variables use
`.size' and `.array[index]'

\begin{mylisting}
\begin{verbatim}
MY_DYNAMIC_ARRAY.size
MY_DYNAMIC_ARRAY.array[i]
\end{verbatim}
\end{mylisting}

To access variables of a model data type use `.variablename'

\begin{mylisting}
\begin{verbatim}
MY_DATA_TYPE.variablename
\end{verbatim}
\end{mylisting}

\subsubsection{Using Model Data Types}

Following is an example of how to use a data type called
\emph{vacancy}.

\begin{mylisting}
\begin{verbatim}
/* Allocate own data type */
vacancy vac;
/* And initialise */
init_vacancy(&vac);
/* Initialise a static array of the data type */
init_vacancy_static_array(&vac_static_array, array_size);
/* Free a data type */
free_vacancy(&vac);
/* Free a static array of a data type */
free_vacancy_static_array(&vac_static_array, array_size);
/* Copy a data type */
copy_vacancy(&vac_from, &vac_to);
/* Copy a static array of a data type */
copy_vacancy_static_array(&vac_static_array_from, &vac_static_array_to, array_size);
\end{verbatim}
\end{mylisting}


\subsubsection{Using Dynamic Arrays}

Dynamic array types are created by adding `\_array' to a data type.
When passing a dynamic array variable to the following functions
place an \& in front of the array.

\begin{mylisting}
\begin{verbatim}
/* Allocate own array */
vacancy_array vacancy_list; /* And
initialise */ init_vacancy_array(&vacancy_list); /* Reset an array
*/ reset_vacancy_array(&vacancy_list); /* Free an array */
free_vacancy_array(&vacancy_list); /* Add an element to the array */
add_vacancy(&vacancy_list, var1, var2, var3); /* Remove an element
at index index */ remove_vacancy(&vacancy_list, index); /* Copy the
array */ copy_vacancy_array(&from_list, &to_list);
\end{verbatim}
\end{mylisting}

\subsection{Sending and receiving messages}

Messages can be read using macros to loop through the incoming message list as per the template below.
Message variables can be accessed using an arrow `->'

\begin{mylisting}
\begin{verbatim}
START_MESSAGENAME_MESSAGE_LOOP
 messagename_message->variablename
FINISH_MESSAGENAME_MESSAGE_LOOP
\end{verbatim}
\end{mylisting}

Messages are sent or added to the message list by
\begin{mylisting}
\begin{verbatim}
 add_messagename_message(var1, var2);
\end{verbatim}
\end{mylisting}
