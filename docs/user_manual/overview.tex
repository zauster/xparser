\documentclass[12pt,a4paper]{article}
\usepackage{t1enc}
\usepackage[latin1]{inputenc}
\usepackage[english]{babel}
\usepackage{url,graphics,lscape}
\usepackage{a4wide}
\usepackage{graphicx}
\usepackage{color}
%\usepackage{fancyheadings}
\usepackage{verbatim}
%\usepackage{xltxtra}

%\vfuzz2pt % Don't report over-full v-boxes if over-edge is small
%\hfuzz2pt % Don't report over-full h-boxes if over-edge is small

%\newenvironment{mylisting}
%{\begin{list}{}{\setlength{\leftmargin}{1em}}\item\small\bfseries}
%{\end{list}}

%\parskip 6pt         % sets spacing between paragraphs
%\parindent 0pt       % sets leading space for paragraphs
%\pagestyle{fancy}
\begin{document}



\title{FLAME
\\Overview}
\author{Simon Coakley}
%\date{17 May 2006}

\maketitle



% \begin{abstract}
% \end{abstract}

%\pagebreak
%\tableofcontents
%\pagebreak

\section{Introduction}

The FLAME framework is an enabling tool to create agent-based models that can
be run on high performance computers (HPCs). Models are created based upon a
model of computation called (extended finite) state machines. By defining
agent-based models in this way the FLAME framework can
automatically generate simulation programs that can run models efficiently on
HPCs.

\section{Model Design}

Extended finite state machines or X-Machines are used to define agents within a
model. 
The basic definition of an
agent would thus, in accordance to the computational model, contain
the following components:
\begin{enumerate}
 \item A finite set of internal states.
 \item A set of transition functions that operate between states.
 \item An internal memory set. In practice, the memory would be a finite set and can be structured in any way required.
 \item A language for sending and receiving messages between other agents.
\end{enumerate}

\begin{equation}\label{streamxmachine}
    X = (\Sigma, \Gamma, Q, M, \Phi, F, q_{0}, m_{0})
\end{equation}
where,
\begin{itemize}
\item $\Sigma$ are the set of input alphabets
\item $\Gamma$ are the set of output alphabets
\item $Q$ denotes the set of states
\item $M$ denotes the variables in the memory.
\item $\Phi$ denotes the set of partial functions $\phi$ that map
and input and memory variable to an output and a change on the
memory variable. The set $\phi$: $\Sigma \times M\ \longrightarrow\
\Gamma\times M$
\item $F$ in the next state transition function, $F : Q \times\phi\longrightarrow
Q$
\item $q_{0}$ is the initial state and $m_{0}$ is the initial memory
of the machine.
\end{itemize}

\subsection{Transition Function}
The transition functions allow the agents to change the state in
which they are in, modifying their behaviour accordingly. These would
require as inputs their current state $s_{1}$, current memory value
$m_{1}$, and the possible arrival of a message that the agent is able to
read, $t_{1}$. Depending on these three values the agent can then
change to another state $s_{2}$, updates the memory to $m_{2}$ and
optionally sends a message, $t_{2}$. Figure
\ref{fig:trans} depicts how the transition function
works within the agent.

% \begin{figure}
% \begin{center}
% \includegraphics*[scale=0.5]{transfn.eps}
% \caption{Transition function} \label{fig:trans}
% \end{center}
% \end{figure}

Some of the transition functions may not depend on the incoming
message. Thus the message would then be represented as:
\begin{equation}\label{msg}
    Message = \{ \emptyset, <data> \}
\end{equation}

These agent transition functions may be expressed in terms of
stochastic rules, thus allowing the multi-agent systems to be termed
as stochastic systems.

\subsubsection{Memory and States}
The difference between the internal set of states and the internal
memory set allows for added flexibility when modelling systems.
There can be agents with one internal state and all the complexity
defined in the memory or equivalently, there could be agents with
a trivial memory with the complexity then bound up in a large state
space. There are good examples of choosing an appropriate balance
between these two as this enables the complexity of the models to be
better managed.

% \begin{figure}
% \begin{center}
% \includegraphics*[width = 4in]{X-Machine_agent.eps}
% \caption{X-Machine agent} \label{fig:xmachine}
% \end{center}
% \end{figure}

Specifying software behaviour have traditionally involved finite state
machines which allow modelling a system in terms of its inputs and outputs.
More abstract system descriptions include UML which has already been proposed as a way to design agent-based models \cite{BAUER:2000,BAUER:2001,HUGET:2002,WEISBUCH:2000} but these techniques lack precise descriptions needed for generating simulation code and for testing.
Testing a system specified as a finite state machine makes it easier for the behaviour to be expressed as a graph
and allow traversals of all possible and impossible executions of the system \footnote{This is similar to branch traversal testing.}. Conventional state machines describe the state-dependent behaviour of a system in terms of its inputs, but this fails to include the effect of data.
X-Machines are an extension to conventional state machines that
include the manipulation of memory as part of the system behaviour,
and thus are a suitable way to specify agents. The advantages of this
approach have been highlighted in Section \ref{xmachine}. Describing a system would thus include the following individual
stages for creating a model:

\begin{itemize}
\item Identifying the system functions
\item Identify the states which impose some order of function execution
\item Identify the input messages and output messages
\item For each state identify the memory as the set of variables that are accessed by outgoing and incoming transition functions
\end{itemize}


%\appendix

%\section{XML DTD}                       % B
%\label{cha_xmldtd}

%\small{{\tt \verbatiminput{xmml.dtd}}}


%\bibliographystyle{alpha}
%\bibliography{EURACE_refs}

\end{document}
