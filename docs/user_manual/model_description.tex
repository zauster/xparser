\section{Model Description}
\label{model_description}

Models descriptions are formatted in XML tag structures to allow
easy human and computer readability.

The DTD (Document Type Definition) of the XML document is currently located
here:

http://eurace.cs.bilgi.edu.tr/XMML.dtd

The start and end of a model file should be formatted as follows:

\begin{mylisting}
\begin{verbatim}
<?xml version="1.0" encoding="ISO-8859-1"?>
<!DOCTYPE xmodel SYSTEM "http://eurace.cs.bilgi.edu.tr/XMML.dtd">
<xmodel version="2">
<name>Model_name</name>
<version>the version</version>
<description>a description</description>
...
</xmodel>
\end{verbatim}
\end{mylisting}

Models can contain:
\begin{itemize}
\item other models (enabled or disabled)
\item \textbf{environment}
\begin{itemize}
\item constant variables
\item function files
\item time units
% \begin{itemize}
% \item name
% \item *** unit
% \item *** period
% \end{itemize}
\item data types
% \begin{itemize}
% \item name
% \item description
% \item variables
% \end{itemize}
\end{itemize}
\item \textbf{agents}
\begin{itemize}
\item name
\item description
\item memory
% *** variables
\item functions
% *** name
% *** description
% *** current state
% *** next state
% *** condition
% *** inputs
% **** filter
% *** outputs
\end{itemize}
\item \textbf{messages}
\begin{itemize}
\item name
\item description
\item variables
\end{itemize}
\end{itemize}

\subsection{Environment}
The environment tag in the model.xml file hosts additional tags for information
that may be required by the parser for efficient simulation of the model.
Following are the tags that can be defined in it.


\subsubsection{Constant Variables}

Constant Variables refers to the global values used in the model. These can me
defined in a separate header file which can then be included in one of the
functions file.

The header file would look as follows:

\begin{mylisting}
\begin{verbatim}
#define <varname> <value>
\end{verbatim}
\end{mylisting}

If this file was saved as a `my\_header.h' file, include this file into one of
the function files so that the compiler knows about these arguments.

\subsubsection{Function Files}

Function files are where you can place source code for the implementation of the
agent functions.

They are included in the compilation script (Makefile) of the produced model.

\begin{mylisting}
\begin{verbatim}
 <functionFiles>
 <file>function_source_code_1.c</file>
 <file>function_source_code_2.c</file>
 </functionFiles>
\end{verbatim}
\end{mylisting}

\subsubsection{Time Rules}\label{timeunit}

Time rules allow the possibility of restricting the functions to
only execute during particular iterations. An iteration refers to
the smallest unit the models are set up on. In EURACE, we are
assuming every iteration to represent one day in the calender.

Time rules can be applied to function conditions instead of a
condition rule and are defined by a time period and a phase. A time
phase is the offset from the start of a period.

A time unit would contain:
\begin{itemize}
\item name - name of the time unit.
\item unit - can contain the iteration or other time units.
\item period - offset by which to skip the unit in every iteration.
\end{itemize}

A time period needs to be defined as a time unit in the environment
of a model. Time units can be described as:

\begin{mylisting}
\begin{verbatim}
<timeUnits>
  <timeUnit>
    <name>daily</name>
    <unit>iteration</unit>
    <period>1</period>
  </timeUnit>

  <timeUnit>
    <name>weekly</name>
    <unit>daily</unit>
    <period>5</period>
  </timeUnit>

  <timeUnit>
    <name>monthly</name>
    <unit>weekly</unit>
    <period>4</period>
  </timeUnit>

  <timeUnit>
    <name>quarterly</name>
    <unit>monthly</unit>
    <period>3</period>
  </timeUnit>

  <timeUnit>
    <name>yearly</name>
    <unit>monthly</unit>
    <period>12</period>
  </timeUnit>

</timeUnits>
\end{verbatim}
\end{mylisting}

A condition can then be added to the function definition to make the
parser aware that this particular function only has to be called at
certain times of the simulation. The condition can be added as
follows:
\begin{mylisting}
\begin{verbatim}
 <condition>
     <time>
     <period>monthly</period>
     <phase>a.day_of_month_to_act</phase>
     </time>
 </condition>
\end{verbatim}
\end{mylisting}

The condition allows the function to run \emph{monthly} at the phase
of \emph{day\_of\_month\_to\_act}. The
\emph{day\_of\_month\_to\_act} is a variable extracted from the
agent memory and is thus defined as
\emph{a.day\_of\_month\_to\_act}.

Refer to section \ref{functioncond} for more details on function
condition definitions.

These rules are then parsed into rule functions and placed in a file
called rules.c

\subsubsection{Data Types}

Data types are user defined data types that can be used in a model.

Data types can contain C data types or other predefined user data types.

\begin{mylisting}
\begin{verbatim}
<dataTypes>

 <dataType>
  <name>Histogram</name>
  <description>ADT Histogram</description>
  <variables>
   <variable><type>double</type><name>prob[30]</name><description></description>
   </variable>
   <variable><type>double</type><name>values[30]</name><description></description>
   </variable>
   <variable><type>double</type><name>max</name><description></description>
   </variable>
  </variables>
 </dataType>

 <dataType>
  <name>Belief</name>
  <description>ADT Belief</description>
  <variables>
   <variable><type>double</type><name>expectedPriceReturns</name><description>
                </description>
   </variable>
   <variable><type>double</type><name>expectedTotalReturns</name><description>
                </description>
   </variable>
   <variable><type>double</type><name>expectedCashFlowYield</name><description>
                </description>
   </variable>
   <variable><type>double</type><name>volatility</name><description></description>
   </variable>
   <variable><type>Histogram</type><name>hist</name><description></description>
   </variable>
  </variables>
 </dataType>

</dataTypes>
\end{verbatim}
\end{mylisting}

In the example above the data type \emph{Belief} contains a variable
of data type \emph{Histogram}

\subsection{Agent}

An agent can be defined as follows:
\begin{mylisting}
\begin{verbatim}
<agents>

  <xagent>
    <name>Firm</name>
    <description></description>
    <memory>
      <variable><type>int</type><name>id</name><description></description></variable>
      <variable><type>int</type><name>region_id</name><description></description>
                    </variable>
      <variable><type>int</type><name>gov_id</name><description></description></variable>
      <variable><type>int</type><name>day_of_month_to_act</name><description>
                    </description></variable>
      <variable><type>double</type><name>payment_account</name><description>
                    </description></variable>
    </memory>
    <functions>
      .....
    </functions>
  </xagent>

  <xagent>
    <name>Household</name>
    <description></description>
    <memory>
      <variable><type>int</type><name>id</name><description></description></variable>
      <variable><type>int</type><name>region_id</name><description></description>
                    </variable>
      <variable><type>int_array</type><name>neighboring_region_ids</name><description>
                    </description></variable>
      <variable><type>int</type><name>gov_id</name><description></description></variable>
      <variable><type>int</type><name>day_of_month_to_act</name><description>
                    </description></variable>
      <variable><type>double</type><name>payment_account</name><description>
                    </description></variable>
    </memory>
    <functions>
      <function>
        <name>Household_read_firing_messages</name>
        <description>The household checks whether is is fired or not</description>
        <currentState>EXIT_FINANCIAL_MARKET</currentState>
        <nextState>01d</nextState>
        <condition><lhs><value>a.employee_firm_id</value></lhs><op>NEQ</op><rhs>
                        <value>-1</value></rhs></condition>
        <inputs>
          <input><messageName>firing</messageName></input>
        </inputs>
      </function>
    </functions>
  </xagent>
</agents>
\end{verbatim}
\end{mylisting}

\subsubsection{Agent Functions}
An agent function contains:
\begin{itemize}
\item name
\item description
\item current state - The current state point the agent is in, in
the branch.
\item next state - The next state point it has to move to, in the
branch.
\item condition - Condition which tells the parser when this
function can be executed.
\item inputs - The messages this function is reading.
\item outputs - The messages this function is sending.
\end{itemize}

\subsubsection{Function Condition and Message Input Filter Rule
Tags}\label{functioncond}

The agent functions can be accompanied by conditions telling the
parser when these functions have to be executed during the
iterations. Condition on the basis of time have been explained in
section \ref{timeunit} but these can also be associated with
comparison rules.

\paragraph{Comparison Rules}

\begin{mylisting}
\begin{verbatim}
<lhs></lhs><op></op><rhs></rhs>
\end{verbatim}
\end{mylisting}

lhs and rhs can be either a value, denoted by value tags:

\begin{mylisting}
\begin{verbatim}
<value></value>
\end{verbatim}
\end{mylisting}

or another rule.

Values can include agent and message memory variables, which are denoted by either:

\begin{mylisting}
\begin{verbatim}
a.agent_var
m.message_var
\end{verbatim}
\end{mylisting}

op can be either comparison functions:

\begin{itemize}
\item EQ -- equal to
\item NEQ -- not equal to
\item LEQ -- less than or equal to
\item GEQ -- greater than or equal to
\item LT -- less then
\item GT -- greater than
\end{itemize}
or logic operators:
\begin{itemize}
\item AND
\item OR
\end{itemize}
the operator NOT is used by placing `not' tags around a rule:
\begin{mylisting}
\begin{verbatim}
<condition>
 <lhs>
  <lhs><value>a.employee_firm_id</value></lhs>
  <op>GT</op>
  <rhs><value>-1</value></rhs>
 </lhs>
 <op>AND</op>
 <rhs>
  <not>
  <lhs><value>a.on_the_job_search</value></lhs>
  <op>EQ</op>
  <rhs><value>0</value></rhs>
  </not>
 </rhs>
</condition>
\end{verbatim}
\end{mylisting}

\paragraph{Message Filter}
Message filters allow the messages to be filtered before being
provided to the function. This allows the messages to be checked
according to a condition before being read by the function. The
message filter can be added as follows:

\begin{mylisting}
\begin{verbatim}
<input>
 <messageName>firing</messageName>
 <filter>
  <lhs><value>a.id</value></lhs>
  <op>EQ</op>
  <rhs><value>m.worker_id</value></rhs>
 </filter>
</input>
\end{verbatim}
\end{mylisting}

Thus in the above example messages will be filtered according to the
message variable \emph{worker\_id} (defined as m.<varname>) to be EQ
(equal) to the agent \emph{id} (defined as a.<varname>).

\subsection{Messages}

The agents are communicating with each other through messages being
sent in the system. The messages being processed have to be defined
in the XML file. All message lists are emptied at the end of the
iteration.

\begin{mylisting}
\begin{verbatim}
<messages>

 <message>
  <name>bank_account_update</name>
  <description>Sent by household. Household informs the bank about the actual
                payment account</description>
  <variables>
    <variable><type>int</type><name>id</name><description></description></variable>
    <variable><type>int</type><name>bank_id</name><description></description></variable>
    <variable><type>double</type><name>payment_account</name>
                <description></description></variable>
  </variables>
 </message>

 <message>
  <name>application_rejection</name>
  <description>Send by firms. Includes the id and the id of the refused applicant.
                </description>
  <variables>
    <variable><type>int</type><name>firm_id</name><description></description></variable>
    <variable><type>int</type><name>worker_id</name><description></description></variable>
  </variables>
 </message>

</messages>
\end{verbatim}
\end{mylisting}
